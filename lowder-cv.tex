\documentclass[10pt]{article}

\usepackage{calc}

\reversemarginpar

\usepackage[paper=a4paper,
            %includefoot, % Uncomment to put page number above margin
            marginparwidth=1.2in,     % Length of section titles
            marginparsep=.05in,       % Space between titles and text
            margin=0.75in,               % 1 inch margins
            includemp]{geometry}

%\usepackage[paper=a4paper,
%            %includefoot, % Uncomment to put page number above margin
%            marginparwidth=30.5mm,    % Length of section titles
%            marginparsep=1.5mm,       % Space between titles and text
%            margin=25mm,              % 25mm margins
%            includemp]{geometry}

\setlength{\parindent}{0in}

\usepackage{paralist}

\usepackage{fancyhdr,lastpage}
\pagestyle{fancy}
%\pagestyle{empty}      % Remove page numbering
\fancyhf{}\renewcommand{\headrulewidth}{0pt}
\fancyfootoffset{\marginparsep+\marginparwidth}
\newlength{\footpageshift}
\setlength{\footpageshift}
          {0.5\textwidth+0.5\marginparsep+0.5\marginparwidth-2in}
\lfoot{\hspace{\footpageshift}%
       \parbox{4in}{\, \hfill %
                    \arabic{page} of \protect\pageref*{LastPage} % +LP
%                    \arabic{page}                               % -LP
                    \hfill \,}}

\usepackage{color,hyperref}
\definecolor{black}{rgb}{0.0,0.0,0.0}
\hypersetup{colorlinks,breaklinks,
            linkcolor=black,urlcolor=black,
            anchorcolor=black,citecolor=black}

\newcommand{\makeheading}[1]%
        {\hspace*{-\marginparsep minus \marginparwidth}%
         \begin{minipage}[t]{\textwidth+\marginparwidth+\marginparsep}%
                {\large \bfseries #1}\\[-0.15\baselineskip]%
                 \rule{\columnwidth}{1pt}%
         \end{minipage}}

\renewcommand{\section}[2]%
        {\pagebreak[2]\vspace{1.3\baselineskip}%
         \phantomsection\addcontentsline{toc}{section}{#1}%
         \hspace{0in}%
         \marginpar{
         \raggedright \scshape #1}#2}

\newenvironment{outerlist}[1][\enskip\textbullet]%
        {\begin{enumerate}[#1]}{\end{enumerate}%
         \vspace{-.6\baselineskip}}

\newenvironment{lonelist}[1][\enskip\textbullet]%
        {\vspace{-\baselineskip}\begin{list}{#1}{%
        \setlength{\partopsep}{0pt}%
        \setlength{\topsep}{0pt}}}
        {\end{list}\vspace{-.6\baselineskip}}

\newenvironment{innerlist}[1][\enskip\textbullet]%
        {\begin{compactenum}[#1]}{\end{compactenum}}

\newcommand{\blankline}{\quad\pagebreak[2]}

\begin{document}
\makeheading{Chris Lowder}

\section{Contact}
\newlength{\rcollength}\setlength{\rcollength}{3.0in}%
%
\begin{tabular}[t]{@{}p{\textwidth-\rcollength}p{\rcollength}}
\href{https://www.dur.ac.uk/mathematical.sciences/}{Department of Mathematical Sciences}
                           & \textit{Mobile:} +44 (0) 7497 356988\\
\href{https://www.dur.ac.uk}{Durham University}
                           & \textit{Office:} ~+44 (0) 191 334 3087\\
Durham, DH1 3LE, United Kingdom    & \textit{E-mail:} \href{mailto:chris.lowder@durham.ac.uk}{chris.lowder@durham.ac.uk}
\end{tabular}

\section{Education}
\href{http://www.montana.edu/}{\textbf{Montana State University}},
Bozeman, Montana, United States
\begin{outerlist}
\setlength\itemsep{0em}
\item[] PhD
	\href{http://www.physics.montana.edu/}
	     {Physics}, June 2015
	     %GPA : 3.95 / 4.00\\
\item[] M.S.
	\href{http://www.physics.montana.edu/}
	     {Physics}, May 2011\\
\end{outerlist}

\href{http://www.gatech.edu/}{\textbf{Georgia Institute of Technology}}, 
Atlanta, Georgia, United States
\begin{outerlist}
\item[] B.S., 
        \href{http://www.physics.gatech.edu/}
             {Physics}, December 2007
	     %GPA : 3.84 / 4.00        
\end{outerlist}

\section{Publications}
\hangindent=0.2cm \href{http://adsabs.harvard.edu/abs/2017SoPh..292...18L}{Lowder, C., Qiu, J., \& Leamon, R. \textit{Coronal Holes and Open Magnetic Flux over Cycles 23 and 24.} SoPh 292, 18 (2017).}

\hangindent=0.4cm \href{http://adsabs.harvard.edu/abs/2014ApJ...783..142L}{Lowder, C., Qiu, J., Leamon, R. \& Liu, Y. \textit{Measurements of EUV Coronal Holes and Open Magnetic Flux.} ApJ 783, 142 (2014).}

\hangindent=0.4cm Lowder, C., Qiu, J., Leamon, R., \& Longcope, D.  \textit{Connecting Coronal Holes and Open Magnetic Field}. (in preparation).

\hangindent=0.4cm Lowder, C., Qiu, J., \& Leamon, R., \textit{Transient Coronal Dimmings and connection to Heliospheric Open Flux}. (in preparation).

\hangindent=0.4cm Lowder, C., Yeates, A., \textit{Magnetic Flux Rope Identification and Characterization from\break Observationally-Driven Solar Coronal Models}. (in preparation).

\section{Selected Conference Proceedings}
\hangindent=0.4cm \textit{Magnetic Flux Rope Identification and Characterization from Observationally-Driven Solar Coronal Models} UK National Astronomy Meeting (2016).

\hangindent=0.4cm \textit{Connecting Coronal Holes and Open Magnetic Field via Numerical Modeling and Observations.} Triennial Earth-Sun Summit  / SPD (2015).

\hangindent=0.4cm \textit{A Comparison of EUV Coronal Hole Measurements and Modeled Open Magnetic Field -or- How I learned to stop worrying and love the potential magnetic field.} GSU Colloquium Series (2014).

\hangindent=0.4cm \textit{Full Surface Automated Coronal Hole Detection and Characterization to Constrain Global Magnetic Field Models}. AAS Meeting 220 (2012).

\hangindent=0.4cm \textit{Transient coronal holes : A statistical study of coronal dimming regions}. The Origin, Evolution, and Diagnosis of Solar Flare Magnetic Fields and Plasmas (2010).

\hangindent=0.4cm \textit{Coronal Mass Ejections : A Study of Structural Evolution and Classification}. AAS Meeting 210 (2007).

\section{Computing}
\textit{Proficient} : Python, NumPy, SciPy, IDL, \LaTeX, OpenMPI, Fortran, Git/GitHub, Glue

\textit{Familiar} : C++, Octave, MATLAB, OpenCL, VisIt

Experience in parallel high performance computing projects and large-scale datasets

%\href{http://www.georgiasouthern.edu/}{\textbf{Georgia Southern University}}, 
%Statesboro, Georgia, United States
%\begin{outerlist}

%\item[] Post-Secondary Options Student / Joint Enrollment
%        \begin{innerlist}
%        \item GPA : 4.00 / 4.00
%        \end{innerlist}
        
%\end{outerlist}

\section{Research Experience}
\href{https://www.dur.ac.uk}{\textbf{Durham University}}, 
Durham, United Kingdom\\
\href{https://www.dur.ac.uk/mathematical.sciences/}{\textbf{Department of Mathematical Sciences}}
\begin{outerlist}
\item[] \textit{Postdoctoral Research Associate}%
        \hfill \textbf{August 2015 to Present}
\begin{innerlist}
  \item Working with \href{http://www.maths.dur.ac.uk/~bmjg46/}{Anthony Yeates} on modelling solar flux rope eruption.
  \item Utilized global non-potential models of the solar magnetic field, magnetic flux ropes are automatically identified and characterized throughout the span of the solar activity cycle.
  \item Developed software routines for managing and visualizing large datasets.
  \item Organized UKMHD 2017 meeting in Durham
  \\
\end{innerlist}

\end{outerlist}

\pagebreak

\href{http://www.montana.edu}{\textbf{Montana State University}}, 
Bozeman, Montana, United States\\
\href{http://www.physics.montana.edu}{\textbf{School of Physics}}
\begin{outerlist}
\item[] \textit{Graduate Research Assistant}%
        \hfill \textbf{August 2009 to August 2015}
\begin{innerlist}
  \item Worked with \href{http://solar.physics.montana.edu/qiuj/}{Dr. Jiong Qiu} and \href{http://solar.physics.montana.edu/leamon/}{Dr. Robert Leamon} in analyzing coronal dimming
  \item Designed automated code to detect and characterize coronal holes from SDO and STEREO EUV data to constrain global models of open magnetic field
  \item Developed flux transport model to study evolution of far-side open magnetic field
  \item Designed and supervised two projects for undergraduate research students as a part of the MSU solar REU program
  \\
\end{innerlist}

\end{outerlist}

\href{http://solar.physics.montana.edu/}{\textbf{Montana State University}}, 
Bozeman, Montana, United States\\
\href{http://solar.physics.montana.edu/}{\textbf{Solar Physics Group}}
\begin{outerlist}
\item[] \textit{NSF Summer REU Undergraduate Researcher}%
        \hfill \textbf{June 2007 to August 2007}
\begin{innerlist}

  \item Improved methods to resolve the 180-degree ambiguity in solar vector \mbox{magnetograms}
  \item Attempted to apply method to high resolution Hinode magnetograms
  \\
\end{innerlist}

\end{outerlist}


\href{http://www.ifa.hawaii.edu/}{\textbf{University of Hawai'i}}, 
Honolulu, Hawai'i United States\\
\href{http://www.ifa.hawaii.edu/}{\textbf{Institute for Astronomy}}
\begin{outerlist}
\item[] \textit{NSF Summer REU Undergraduate Researcher}%
        \hfill \textbf{May 2006 to August 2006}
\begin{innerlist}
  \item Analysis of CMEs utilizing SOHO data for Dr. Shadia Habbal and Dr. Huw Morgan
  \item Observational experience and interaction with astronomers at Mauna Kea \mbox{observatories} on the IRTF, Caltech CSO, and the UH 88'' \\
\end{innerlist}

\end{outerlist}

\section{Teaching Experience}
\href{http://www.gatech.edu}{\textbf{Georgia Institute of Technology}}, 
Atlanta, Georgia, United States\\
\href{http://www.physics.gatech.edu}{\textbf{School of Physics}}
\begin{outerlist}
\item[] \textit{Physics I / II Graduate Teaching Assistant}%
        \hfill \textbf{August 2008 to May 2009}
\begin{innerlist}

  \item Designed and marked problem sets covering mechanics and electromagnetism
  \item Engaged students in problem solving methods not directly addressed in lecture
  \\
\end{innerlist}

\end{outerlist}

\href{http://georgiasouthern.edu/physics/}{\textbf{Georgia Southern University}}, 
Statesboro, Georgia, United States\\
\href{http://georgiasouthern.edu/physics/}{\textbf{Department of Physics}}

\begin{outerlist}
\item[] \textit{Physics I / II Lab Teaching Assistant}%
        \hfill \textbf{May 2008 to July 2008}
\begin{innerlist}
  \item Maintained lab equipment and helped to integrate the lecture and lab experience
  \item Graded work assignments and assisted with in-class assignments
\end{innerlist}

\end{outerlist}


\begin{outerlist}
\item[] \textit{Astronomy Laboratory Instructor}%
        \hfill \textbf{January 2008 to May 2008}
\begin{innerlist}
  \item Engaged students in aspects of theory and observations in astronomy
  \item Modernized course content and implemented new observational activities
\end{innerlist}

\end{outerlist}

\begin{outerlist}
\item[] \textit{Planetarium Lecturer}%
        \hfill \textbf{January 2008 to May 2008}
\begin{innerlist}
  \item Provided free planetarium shows to grade school level groups
  \item Organized workshop sessions to train grade-school earth science teachers
  \\
\end{innerlist}

\end{outerlist}

\href{http://www.physics.gatech.edu/}{\textbf{Georgia Institute of Technology}}, 
Atlanta, Georgia, United States\\
\href{http://www.physics.gatech.edu/}{\textbf{School of Physics}}
\begin{outerlist}
\item[] \textit{Physics II Laboratory Teaching Assistant}%
        \hfill \textbf{September 2007 to December 2007}
\begin{innerlist}
  \item Setup and conducted a physics II lab session
  \item Instructed students and graded the resulting labwork
  \\
\end{innerlist}

\end{outerlist}

\section{Honors}
\href{http://www.vsp.ucar.edu/Heliophysics/summer-about-over.shtml}{Living with a Star Heliophysics Summer School} (Summer 2015)

Triennial Earth-Sun Summit Student Travel Grant (2015)

Living with a Star Portland Meeting - Best Student Poster (2014)

\href{http://spd.aas.org/studentships.html}{SPD Studentship Travel Award} (2012)

National Merit Scholar (2004)

Georgia Governor's Scholar (2002)

\href{http://www.gatech.edu}{Georgia Institute of Technology}
\begin{innerlist}
\item Faculty Honors (Fall 2004, Spring and Fall 2006)
\item Dean's List (Spring and Fall 2005)
\end{innerlist}

%\href{http://www.gatech.edu}{Georgia Institute of Technology}
%\begin{innerlist}
%\item Faculty Honors (Semester GPA of 4.0), Fall 2004, Spring and Fall 2006
%\item Dean's List (Semester GPA of 3.0), Spring and Fall 2005
%\end{innerlist}

\section{Outreach}
\hangindent=0.4cm Durham University School Science Festival - Organizing activity on solar magnetism

Peaks and Potentials - Taught summer student workshop series on solar physics

MSU Astronomy Day - Organized solar physics exhibit

Montana Science Olympiad - Designed state astronomy event

Georgia Southern Planetarium - Created and presented planetarium show content

%\href{http://astronomyclub.gatech.edu/}{Georgia Tech Astronomy Club} - President


\end{document}
